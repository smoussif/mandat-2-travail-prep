\documentclass[11pt,letterpaper]{article}
\usepackage{listings}
\usepackage{media9,graphicx}
\usepackage{microtype}
\usepackage[implicit=false]{hyperref}
\usepackage[top=2cm, bottom=2cm, left=2cm, right=2cm, columnsep=20pt]{geometry}
\usepackage{graphicx} % Required for inserting images
\usepackage{ragged2e} 
\usepackage[french]{babel}
\usepackage[backend=bibtex]{biblatex}
\usepackage[dvipsnames]{xcolor}
\usepackage{color,soul}   
\usepackage[utf8]{inputenc}
\usepackage[export]{adjustbox}
\usepackage{fancyhdr}
\usepackage{mwe}
\usepackage{caption}
\usepackage{amsmath}
\usepackage{float}
\usepackage{here}
\usepackage{parskip}
\usepackage{subfigure}
\usepackage{pdfpages}
\usepackage{multicol}
\usepackage{biblatex}
\usepackage{amsmath,amssymb}

\newcommand*\Laplace{\mathop{}\!\mathbin\bigtriangleup}
\newcommand*\DAlambert{\mathop{}\!\mathbin\Box}
\addbibresource{bibli.bib}

\lstset{
  language=Latex, % Set the language to Python
  basicstyle=\ttfamily\small,
  keywordstyle=\color{blue},
  commentstyle=\color{teal},
  stringstyle=\color{red},
  numbers=left,
  numberstyle=\tiny,
  stepnumber=1,
  numbersep=5pt,
  breaklines=true,
}

\DeclareCaptionLabelFormat{mycaptionlabel}{#1 #2}
\captionsetup[table]{labelsep=colon}
\captionsetup[figure]{labelsep=colon}
\captionsetup{labelformat=mycaptionlabel}
\captionsetup[table]{name={Tableau }}
\captionsetup[figure]{name={Figure }}

\begin{document}
\include{presentation}
\twocolumn[]

\section{Introduction}


%Contexte, présentation du problème à résoudre et de la solution proposée, résumé du document : méthodologie utilisée, résultats principaux obtenus, impact sur le mandat. Attention, ici le problème à résoudre n’est pas le mandat lui-même, mais bien le problème posé spécifiquement pour le travail préparatoire.
\section{Méthodes}

%Transformer cette question dans un langage mathématique, décrire les méthodes de simulations
\section{Résultats}

%Résultats choisis, obtenus à l’aide des Méthodes et permettant d’informer la Discussion. Attention, ici, les résultats doivent être décrits de façon factuelle, mais ne doivent pas être discutés. Typiquement des figures et des tableaux seront utilisés et le texte décrira de façon neutre les aspects importants des tableaux et figures à remarquer. Plus d’attention devra être portée aux variables qui ont un effet (positif ou négatif) et moins d’attention sera portée aux variables n’ayant pas ou peu d’effet.
\section{Discussion}


%Présentation et interprétation des principes, des relations et des généralisations pouvant être déduits des résultats. Description des exceptions à ces généralisations si présentes. Description des limitations du modèle. Recommandation convaincante sur la démarche à suivre pour remplir le mandat.
\end{document}
